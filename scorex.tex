\documentclass[]{report}   % list options between brackets


%% The amssymb package provides various useful mathematical symbols
\usepackage{amssymb}
%% The amsthm package provides extended theorem environments
\usepackage{amsthm}
\usepackage{amsmath}

\usepackage{listings}


\newtheorem{axiom}{Axiom}

\newtheorem{proposition}{Proposition}


% type user-defined commands here

\begin{document}

\lstset{language=Scala}

\title{On Blockchains And Scorex}   % type title between braces
\author{Alexander Chepurnoy}         % type author(s) between braces
\date{June, 2016}    % type date between braces
\maketitle

% \begin{abstract}  
% \end{abstract}

\chapter{Introduction}

This is a guide to the blockchain data structure and digital currencies built on top of it. Unlike previous guides we do not start with an example of Alice sending a signed note to Bob. Instead, we describing data structures and a process of building them. The guide is made for people understanding basic data structures like sets and append logs. No cryptography knowledge is needed.

We will explain cryptography in very layman terms. For good explanation of cryptography primitives and protocols please refer to the foundational book of~\cite{katz2014introduction}. 

We will describe basic concepts with code snippets in Scala language. No prior knowledge of the Scala language is required.

\section{Other Introductions}

	[TODO: review of other tutorials]


\chapter{Cryptocurrency Design}             % chapter 1
\section{Introduction}     % section 1.1

Blockchain is a prefix-immutable append log of non-conflicting authenticated events happen in a decentralized peer-to-peer network. Simple, eh? \textit{But what all this actually means?}

Simply said, we have peers do not trust each other, and there is no any trusted party, only a strict protocol peers need to follow. Peers are issuing authenticated(e.g. signed) events of some semantics. For example, they are signing payment transactions. Or they are registering \(name \rightarrow value\) correspondences in a shared database(certificates, domains). ``Prefix-immutable append log'' means all the peers following are agree on append log which prefix is immutable(with non-negligible probability). That is, if we cut suffix of some length from an log a peer holds for each peer, results will be the same. And events in the ordered log must be non-conflicting in order to have consistent history.

Consider Bitcoin as an example. Peers are holding money in form of algorithmically issued tokens. They do not trust each other, and do not to seek for a trusted mediator. Instead, they are running a Bitcoin protocol which build prefix-immutable append log containing token transfers. Last few versions of the log are considered as potentially unstable, but before that the history is trusted by anyone. History is consistent, so, for example, if Alice had sent all her tokens to Bob, she can't send anything after that and before receiving tokens from other party.

\section{Transactional Layer}

\subsection{Minimal State}     % section 1.1
	Consider a transaction arrived at a node. The node is doing following on receiving it:

    \begin{enumerate}
		\item Checks whether a transaction is valid

		\item Apply it if so
    \end{enumerate}

	Intuitively, there are some stateless checks, e.g. whether a signature for a transaction is valid, whether amount of tokens to transfer is non-negative, but also there are stateful ones. For example, if Alice is sending tokens to Bob, a node must be sure Alice has enough funds in order to make a payment.

	So a node needs to store some state in order to validate incoming transactions. There is some greatest lower bound for a type of a state for the whole network. That is, there is some \textit{minimal state} representation enough to validate an arbitrary transaction while removing any element from the representation eliminating this property. So all the nodes share this minimal state but a node could also store some additional information. 

	By applying transaction a minimal state is being modified. It should be impossible to apply a transaction already processed. 

	For many reasons almost all cryptocurrencies of today are packing transactions into \textit{blocks}. We can think about a block as of \textit{atomic batch state update}. 

    [TODO: Block header - tx part]


	We can state some axioms here.

	\begin{axiom}
	 There is some initial state hard-coded into each node. Further we name it \textit{genesis state}.
	\end{axiom}

    \begin{axiom} 
     Validation and application of a transaction are deterministic procedures. All the honest nodes follow the same rules. 
    \end{axiom}

	\begin{axiom} 
	 Validation and application of a transactional metadata are deterministic. All the honest nodes follow the same rules. 
	\end{axiom}

	\begin{proposition} If the same sequence of blocks is applied to the genesis state for two different nodes, then the resulting minimal state will be the same.
	\end{proposition}	
	\begin{proof} Consider the nodes have the same minimal state and applying the same block to it. \end{proof}


\subsection{Bitcoin}

In Bitcoin a transaction contains multiple \textit{inputs} and \textit{outputs}. Inputs are connected with outputs of transactions previously applied to a state, and the outputs must not be spent yet. That is, the outputs do not have connections from transactions previously 
applied to a state. Thus an output could be spent as whole only and so we can consider a set of unspent outputs as a minimal state.

How to spend an output? In Bitcoin it contains a script in a stack-based language. Input also contains a script. Then an input could spend an output if a combined script of inputs' and then outputs' could be executed and results in non-zero top stack item.

[TODO: example]

\subsection{Boxes, Propositions and Proofs}

Abstracting the Bitcoin-like model, a minimal state could be represented as a set of \textit{closed boxes} of size \(n_S\). Each box has a value associated with it. A transaction opens \(n_k\) boxes and also creates \(n_b\) new closed boxes. The resulting state set has the size of \(n_S-n_k+n_b\) after applying the transaction to it. 

How to open a box? We can protect a box with a script in Bitcoin language. Or we can put a public key into closed box and in order to open it with a signature verifiable with the public key~(we will consider details further). To describe these approaches as well as many others possible in a general way, we say a box is protected by a \textit{proposition} of some kind, and in order to open it, a \textit{proof} of the same kind must be provided. There are some tricky details we will discuss further.

[TODO: examples and ]

\subsection{Nxt and Ethereum}	

Having a Bitcoin wallet, you can be goggled by complexity of boxes and propositions in form of stack-based scripts. Why not to just have account and token transfers between them? 

Actually some of Bitcoin successors walked this path. For example, Nxt has a dedicated notion of accounts. An account is associated with its public key. Then a just transaction transfers tokens from one account to another and needed to be signed by a sender. For such a system stateful verification needs for a minimal state in form of table holding a correspondence between accounts and their balances.

However, with such a design we have a problem. Let's describe it with an example. Alice has 50 tokens at some moment of time. At this moment she issues a signed transaction to pay 5 tokens to Bob. A node can validate the transaction and found it valid, and so applicable. After the application Alice has 45 tokens. But how to prevent second application of the transaction? Our minimal state representation seems to be flawed.

Ethereum solves the problem by modifying minimal state representation adding ``nonce'' value to it. That is, minimal state is not about (public key -> balance) correspondence anymore, but (public key -> (nonce, balance)). Transaction contains nonce value \(txnonce\) as well, and transaction is valid and so applicable only if \(txnonce = nonce + 1\). By application, \(nonce := txnonce\). 

Unlike Bitcoin, Ethereum sets strict order of transactions issued by an account. In Bitcoin, transactions could be applied in any order, 
if they are spending non-overlapping sets of outputs, and input of one transaction does not spend an output of another. In Ethereum, order of transaction is set by nonce values. 

\subsection{Transactional Metadata}

[TODO: Merkle tree / authenticated data structures explanation]

Along with transactions, we can put some aggregated data about them. For example. in Bitcoin's block a root hash of a Merkle tree for the transactions in block is put into the block. That way it is possible for nodes in a network to exchange not full blocks but \textit{blockheaders}. A \textit{blockheader} is a block without its transactions. By including transactions root hash into the blockheader it is possible to have it spread around a network and be sure it is impossible to show transactions set other that was included. 


\section{Consensus}

We have proven(in the proposition~[?]) that if the same sequence of blocks carrying transactions is applied to the same genesis state for two node then they will have the same state. It is exactly what do we want to achieve, but how to have the same sequence of blocks for all the nodes? 

In the first place, we can achieve this only for nodes willing to achieve this by following some protocol strictly. We refer to such nodes as to \textit{honest} nodes, and to the protocol as to \textit{consensus protocol}. If nodes are not following the protocol we call them \textit{byzantine} nodes. A byzantine node could be malicious, but also it could be not able to follow the consensus protocol because of software bugs, problems with connectivity, misleading information sent from outer world etc. 

[validity, agreement, termination]

Computer Science studies consensus protocols since early 1980s. A lot of interesting results were generated in this field. For example, it is impossible to achieve consensus using a deterministic procedure for a set of nodes if they are exchanging messages asynchronously and a single process could fail(Fischer-Lynch-Paterson theorem~\cite{}). 

Consensus in open networks, so with unknown number of participants, is pretty new and very hard question. 

\begin{enumerate}
\item{Validity}
\item{Agreement}
\item{Termination}
\end{enumerate}

For a blockchain consensus protocols, we can state following properties:

\begin{enumerate}
\item{Consistency(or Prefix immutability)} - for two honest nodes the probability to have different prefixes after cutting last \(k\) blocks should go down with \(k\) and be negligible after some value. The good option is to have the probability going down exponentially with \(k\).
\item{Chain Quality} - a party having \(x\%\) of voting power should produce no more than \((\alpha \cdot) x \%\) blocks in a long run, where \(\alpha\) is constant.
\item{Chain Growth} - over time blockchain should always grow. No one is interested in a structure with possibility to stuck. 
\end{enumerate}


\subsection{Proof-of-Work}

\subsection{Proof-of-Stake}

\subsection{History}       % subsection 1.1.1


\section{Peer-to-Peer Network}	

Blockchain is maintained in a peer-to-peer network. For simplicity we are starting with a network where all the nodes 

\section{Incentives}



\section{Complications}

fully prunable outputs 
ZCash

\section{Conclusion}


\section{Further Reading}

Proof-of-Work and blockchain were introduced in the foundational Bitcoin whitepaper~\cite{Nakamoto2008}.

Overview of Bitcoin P2P layer along with description of possible Eclipse attacks(partly fixed to the moment) against it could be found in~\cite{heilman2015eclipse}


\chapter{Scorex}           % chapter 2
\section{Introduction}     % section 2.1

Scorex is a modular blockchain core framework. It supports definitions given in the previous chapter in form of Scala code. What do you need to do in order to build something on Scorex, is to implement all the abstract interfaces, or just some of them and use ready modules for missing parts.

\subsection{*Scala Language}         % subsection 2.1.1

We will describe some concepts of Scala language in sections started with ``*''. If you already a Scala developer, you can miss the sections. Experience with programming languages(say, Java, C++, OCaml or Rust) is needed, as we will explain Scala features used in code snippets provided very quickly. For a good introduction into the Scala language, please refer to~\cite{?}.

Scala is functional, modular and also object-oriented language. Such a mix of concepts means different developers can follow very different styles. 

[TODO: quick description]

\section{Transactional Layer}

Transactional layers describes blockchain semantics. That is, what is a minimal state enough to validate incoming transactions, what is transaction and how processing of it affects the state, how transactions are protected from being spent by non-allowed party, how wallet is organized etc

\subsection{*Scala: Traits and Type Parameters}

The basic piece in a sufficiently large Scala codebase is a \textit{trait}. Trait is an abstract(so non-instantiable) interface describing functions and values a concrete implementation(class or object) must provide to its users. We can also think about trait as of type and also a parametrized module. A trait could be parametrized not only by values, but also by types. 

Quick example: 
[TODO: example]

\subsection{Propositions and Proofs}

In the first place, we are getting into a mechanism to protect binary objects, e.g. transaction outputs, from non-permissioned access. We protect an object with a \textit{proposition}. Then in order to make an action with an object it is needed to provide a \textit{proof} for its proposition. 

Proposition is very abstract concept. The only property we require from it is to be serialized into bytes. In form of Scala code it is described as:

\begin{lstlisting}
trait Proposition extends BytesSerializable
\end{lstlisting}

In most useful scenarios a proposition should be addressable. That is, it could be addressed by some identifier, or identifiers.

A proof is an object which could satisfy a proposition given an additional input, namely a \textit{message}(e.g. transaction bytes). As well as a proposition, a proof could be serialized into bytes. 

\begin{lstlisting}
trait Proof[P <: Proposition] extends BytesSerializable {
  def isValid(proposition: P, message: Array[Byte]): Boolean
  ...
}
\end{lstlisting}


\subsection{Box}

A box is a minimal state element. An unspent output in Bitcoin is a box. An account in certain state in Nxt or Ethereum is also a box. Basically, a box is about some value(how many system tokens are associated with it), and some proposition which protects a box from being spent by anyone but a party(or parties) knowing how to satisfy the proposition.

\begin{lstlisting}
trait Box[P <: Proposition] extends BytesSerializable {
  val proposition: P

  val value: Long

  val id: Array[Byte]
}
\end{lstlisting}


\subsection{*Scala: Sum Types and Try}

\subsection{A Transaction and Minimal State}

As it was shown in the Chapter 1[TODO: link], a transaction and a minimal state could be defined via each other: a transaction is a state modifier, which also could be valid or not against a state, and applicable if and only if it is valid. A minimal state is a data structure which deterministically defines whether an arbitrary transaction is valid and so applicable to it or not. 

\begin{lstlisting}
abstract class Transaction[P <: Proposition, TX <: Transaction[P, TX]] extends BytesSerializable with JsonSerializable {
  def validate(state: MinimalState[P, TX]): Try[Unit]
  def changes(state: MinimalState[P, TX]): Try[StateChanges[P]]

  ...
}
\end{lstlisting}

We are defining functional interface for minimal state below:

\begin{lstlisting}
trait MinimalState[P <: Proposition, TX <: Transaction[P, TX]] {
  def version: Int

  def isValid(tx: TX): Boolean = tx.validate(this).isSuccess

  def processBlock(block: Block[P, _, _]): Try[Unit]
  def rollbackTo(height: Int): Try[Unit]

  def closedBox(boxId: Array[Byte]): Option[Box[P]]

  ...
}
\end{lstlisting}

So a minimal state knows its current version, can apply a block, and also rollback to a previous version. To validate a transaction, it just calls \textit{validate} function of a transaction and checks its result.

\subsection{Wallet}

[TODO:]

\subsection{Memory Pool}

\begin{lstlisting}
trait UnconfirmedTransactionsDatabase[TX <: Transaction[_, TX], TData <: TransactionalData[TX]] {
  def putIfNew(tx: TX): Boolean

  def all(): Seq[TX]

  def getById(id: Array[Byte]): Option[TX]

  def packUnconfirmed(): TData

  def clearFromUnconfirmed(data: TData): Unit

  def onNewOffchainTransaction(transaction: TX): Unit

  def remove(tx: TX)
  
}
\end{lstlisting}

\subsection{Transactional Block Data}

\begin{lstlisting}
trait TransactionalData[TX <: Transaction[_, TX]] extends BytesSerializable with JsonSerializable {
  val mbTransactions: Option[Traversable[TX]]

  val headerOnly = mbTransactions.isDefined
  ....
}
\end{lstlisting}

\subsection{Transactional Module}



\section{Consensus Layer}

\subsection{Block}

\begin{lstlisting}
class Block[P <: Proposition, TData <: TransactionalData[_ <: Transaction[P, _]], CData <: ConsensusData](
                                                                                      val version: Byte,
                                                                                      val timestamp: Long,
                                                                                      val consensusData: CData,
                                                                                      val transactionalData: TData)
\end{lstlisting}

\subsection{History and Blockchain}

\section{Network Layer}

Network layer in Scorex is simpler than in Bitcoin or Nxt. 

\subsection{Peer Discovery}

\subsection{Broadcasting Strategies}

\subsection{View Synchronizing}

\section{Ready Modules}
There are few modules already implemented.

\subsection{Proof-of-Stake}
Proof-of-Stake module contains implementations of two Proof-of-Stake consensus algorithms. 

\subsection{Simple Transactions Module}

Simplest Transactions Module contains an implementation of transactional module with only one kind of transactions, just tokens transfers from one public key to another.

\subsection{Permacoin Implementation Module}

The module contains an implementation of Permacoin consensus protocol~\cite{miller2014permacoin}.

\section{Conclusion}

\bibliographystyle{}
\bibliography{sources.bib}

\end{document}